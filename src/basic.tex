%%%%%%%%%%%%%%%%%%%%%%%%%%%%%%%%%%%%%%%%

\begin{edXchapter}{Basic examples}

%%%%%%%%%%%%%%%%%%%%%%%%%%%%%%%%%%%%%%%%%%%%%%%%%%%%%%%%%%%%%%%%%%%%%%%%%%%%%

\begin{edXsection}{Basic example problems}

\begin{edXvertical}

\begin{edXproblem}{Option response}

This is a sample problem, which is worth 10 points.

Give the correct python {\tt type} for the following expressions.  Select {\tt noneType} if the expression is illegal.

\begin{itemize}
\item \edXinline{\tt 3~~~}   \edXabox{expect="int" options="noneType","int","float" type="option" inline="1"}
\item \edXinline{\tt 5.2~~~} \edXabox{expect="float" options="noneType","int","float" type="option" inline="1"}
\item \edXinline{\tt 3/2~~~} \edXabox{expect="int" options="noneType","int","float" type="option" inline="1"}
\item \edXinline{\tt 1+[]~~~} \edXabox{expect="noneType" options="noneType","int","float" type="option" inline="1"}
\end{itemize}

\edXgitlink{\giturl}{Source TeX}

\end{edXproblem}

%%%%%%%%%%%%%%%%%%%%%%%%%%%%%%%%%%%%%%%%

\begin{edXproblem}{String response}

What state is Detroit in?

\edXabox{expect="Michigan" type="string"}

\begin{edXsolution}

Explanations can also be provided inside:
\begin{verbatim}
\begin{edXsolution}
... 
\end{edXsolution}
\end{verbatim}

\end{edXsolution}

\edXgitlink{\giturl}{Source TeX}
\end{edXproblem}


\end{edXvertical}

%%%%%%%%%%%%%%%%%%%%%%%%%%%%%%%%%%%%%%%%%%%%%%%%%%

\begin{edXvertical}


\begin{edXproblem}{Multiple choice single answer}{}

What color is the sky on a clear sunny day?

\begin{edXscript}

aset = [ 'Red',
         'Green',
         'Blue',
         'Black',
      ]

(ans1,ans2,ans3,ans4) = aset

\end{edXscript}

\edXabox{type="multichoice" 
  expect="\$ans3"
  options="\$ans1","\$ans2","\$ans3","\$ans4" }

\edXgitlink{\giturl}{Source TeX}
\end{edXproblem}


\begin{edXproblem}{Multiple choice multiple answers}{}

What are the best computer programming language?  Choose {\em all} which apply:

\begin{edXscript}

aset = [ 'Cobol',
         'Pascal',
         'Python',
         'C++',
         'Clu',
         'Forth',
      ]

(ans1,ans2,ans3,ans4,ans5,ans6) = aset

\end{edXscript}

\edXabox{type="multichoice" 
  expect="\$ans3","\$ans4"
  options="\$ans1","\$ans2","\$ans3","\$ans4","\$ans5","\$ans6"
 }

\edXgitlink{\giturl}{Source TeX}
\end{edXproblem}

\end{edXvertical}

%%%%%%%%%%%%%%%%%%%%%%%%%%%%%%%%%%%%%%%%%%%%%%%%%%

\begin{edXvertical}

%%%%%%%%%%%%%%%%%%%%%%%%%%%%%%%%%%%%%%%%

\begin{edXproblem}{Numerical response}

\section{Example of numerical response}  

What is the numerical value of $\pi$?

\edXabox{expect="3.14159" type="numerical" tolerance='0.01' }

\edXgitlink{\giturl}{Source TeX}
\end{edXproblem}


\begin{edXproblem}{Numerical response with inline labels}{}

Numerical response questions can have inline labels, both before and after the input box:

\begin{itemize}

\item  

    \edXinline{$^2P_{3/2}~~F=3$ $~~~~~ g= $ } \edXabox{expect="0.66666" type="numerical" tolerance='3\%' inline='1' }
    \edXinline{Hz}

\item  

    \edXinline{$^2S_{1/2}~~F=2$ $~~~~~ g= $ } \edXabox{expect="0.5" type="numerical" tolerance='3\%' inline='1' }
    \edXinline{Joules/sec}

\end{itemize}

\begin{edXsolution}

For the stretched states the formula is unnecessary: all the angular momenta are
then aligned with each other and their magnetic moments just add. 

\end{edXsolution}

\edXgitlink{\giturl}{Source TeX}
\end{edXproblem}

\end{edXvertical}

%%%%%%%%%%%%%%%%%%%%%%%%%%%%%%%%%%%%%%%%%%%%%%%%%%

\begin{edXvertical}

%%%%%%%%%%%%%%%%%%%%%%%%%%%%%%%%%%%%%%%%

\begin{edXproblem}{Custom response}

\section{Example of custom response}  

This problem demonstrates the use of a custom python script used for
checking the answer.

\begin{edXscript}

def sumtest(expect,ans):
    try:
        (a1,a2) = map(float,ans)
        return (a1+a2)==10
    except Exception as err:
        return {'ok': False, 'msg': 'Sorry, cannot evaluate your input ' + str(ans)}

\end{edXscript}

Enter two numbers which add up to 10:

\edXabox{expect=""
  type="custom"
  answers="1,9"
  prompts="x = ","y = "
  cfn="sumtest"
  inline="1" }%

\edXgitlink{\giturl}{Source TeX}
\end{edXproblem}

\end{edXvertical}

%%%%%%%%%%%%%%%%%%%%%%%%%%%%%%%%%%%%%%%%%%%%%%%%%%

\begin{edXvertical}

%%%%%%%%%%%%%%%%%%%%%%%%%%%%%%%%%%%%%%%%

\begin{edXproblem}{Formula response}

This problem demonstrates the use of a the ``formula response'' problem.

Let $f(x) = a x^2 + bx + c$.  Assume the coefficients $a$, $b$, $c$
are such that $f$ has two distinct real roots. Give an equation for
$\lambda$, the larger of the two roots, such that $f(\lambda)=0$.''

Use {\tt ^} for exponentiation, e.g. {\tt x^2} denotes $x^2$.
Explicitly include multiplication using ${\tt *}$, e.g. ${\tt x*y}$ is
$xy$.  Standard mathematical functions may be employed, e.g. ${\tt sin(x)}$,
${\tt sqrt(x)}$, etc.

% note that the sampling range must be carefully chosen such that
% the argument of the sqrt does not go negative, for the default random
% numerical sampling checker to work.

\edXinline{$\lambda =$ }
\edXabox{expect="(-b + sqrt(b^2-4*a*c))/(2*a)" type="formula"
  samples="a,b,c@1,16,1:3,20,3#50" size="60" tolerance='0.01' inline='1'
  math="1" feqin="1" }%

\edXgitlink{\giturl}{Source TeX}
\end{edXproblem}

\end{edXvertical}

\end{edXsection}

%%%%%%%%%%%%%%%%%%%%%%%%%%%%%%%%%%%%%%%%%%%%%%%%%%%%%%%%%%%%%%%%%%%%%%%%%%%%%

\begin{edXsection}{Video and text}

\begin{edXvertical}

\begin{edXtext}{Sample show-hide text section}

{\LARGE Sample show-hide text section}

Pieces of text (and images) can be put inside a ``showhide'' section, which the user can hide or show via a click.

\begin{edXshowhide}{ps4starkket}{Hints and instructions for entering expressions}

Note that rotations leave points on the axis of rotation unmoved.  For
a point $(\theta, \phi)$ specified in polar coordinates on the surface
of the \href{http://en.wikipedia.org/wiki/Bloch_sphere}{Bloch sphere},
the corresponding two-level quantum state is
\bea
	|\psi\> = \cos\frac{\theta}{2} |a\> +
        e^{i\phi}\sin\frac{\theta}{2} |b\>
\,.
\eea

Enter the state using the vertical bar ${\tt |}$ and greater-than
symbol ${\tt >}$ to delineate a ``ket'' and enter $\omega_0$ as ${\tt omega\_0}$ as usual.

Expressions like ${\tt (2*|a> + |b>)/sqrt(3)}$ are legal input
(remember to include ${\tt *}$ to denote multiplication, for
coefficients in front of kets).

Standard mathematical functions may be employed, e.g. ${\tt sin(x)}$,
${\tt sqrt(x)}$, ${\tt arccos(x)}$,  ${\tt arctan(x)}$, etc.  

\end{edXshowhide}

\end{edXvertical}

\begin{edXvertical}

\edXxml{<video url_name="week0_intro" display_name="Introduction to 6.SFMx" youtube="1.0:VfOMvuHcJi0" />}

\end{edXvertical}

\end{edXsection}

%%%%%%%%%%%%%%%%%%%%%%%%%%%%%%%%%%%%%%%%%%%%%%%%%%%%%%%%%%%%%%%%%%%%%%%%%%%%%
\end{edXchapter}
