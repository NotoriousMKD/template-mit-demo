% 
% example file for latex2dnd
% this version uses formula checking
%


\documentclass{article}
% \usepackage{amsmath}
\input{latex2dnd}

%%%%%%%%%%%%%%%%%%%%%%%%%%%%%%%%%%%%%%%%%%%%%%%%%%%%%%%%%%%%%%%%%%%%%%%%%%%%%

\begin{document}

% define drag-drop labels

\DDlabel{G}{$G$}
\DDlabel{m1}{$m_1$}
\DDlabel{m2}{$m_2$}
\DDlabel{d}{$d$}
\DDlabel{d2}{$d^2$}

% shorthand macro to make all boxes the same size (6 by 4)
\newcommand\DDB[2]{\DDbox{#1}{6ex}{4ex}{#2}}

% the formula with boxes
$$F = \frac{\DDB{1}{G} \DDB{2}{m1}\DDB{3}{m2}}{\DDB{4}{d2}}$$

% the formula to use for correctness checking, the samples, and the expected answer
% place target_id names inside square brackets
\DDformula{ [1] * [2] * [3] / [4] }{ G,m_1,m_2,d@1,1,1,1:20,20,20,20\#40 }{G*m_1*m_2/d^2}{Try checking units}

% output labels, with fixed box heights
\writeDDlabels[4.3ex]

\end{document}

