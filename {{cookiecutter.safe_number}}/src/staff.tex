%%%%%%%%%%%%%%%%%%%%%%%%%%%%%%%%%%%%%%%%


\begin{edXchapter}{Staff Content}

\begin{edXsection}{Staff Test}

\begin{edXtext}{Staff Test Overview}


{\bf Due date: Friday, 12/6}

\section{Material covered}

This problem set covers the following lectures and reading.  Reading
is from the course text: ``The Physics of Energy'' by Jaffe and Taylor
(preliminary draft).

\begin{itemize}
\item Lecture 32 (11/22):  {\sl Fossil Fuels II}
   (POE  \S 33)
\item Lecture 33 (11/25):  {\sl  Biofuels}
   (POE  \S 26)
\item Lecture 34 (11/27):  {\sl Energy storage}
   (POE  \S 36)
\item Lecture 35 (12/2):  {\sl The electric grid}
   (POE  \S 37)
\item Lecture 36 (12/4):  {\sl Energy and Climate I}
   (POE  \S 34)
\end{itemize}

This problem set has two parts.

The online assignment consists of 6 problems, due 12/6.
 
The written part of the assignment can be found \href{http://physicsofenergy.mit.edu/students/problems.php}{here}.

\end{edXtext}

\end{edXsection}


%%%%%%%%%%%%%%%%%%%%%%%%%%%%%%%%%%%%%%%%%%%%%%%%%%%%%%%%%%%%%%%%%%%%%%%%%%%%%

\begin{edXsection}{Staff Problems I}

\begin{edXvertical}

%%%%%%%%%%%%%%%%%%%%%%%%%%%%%%%%%%%%%%%%
\begin{edXproblem}{Staff Chapter 33 problem 13a}{attempts=5}

\begin{edXscript}
expected=[ 177 ]
\end{edXscript}

\edXinclude{XML/magnitude_hint2.xml}

How does the \href{http://en.wikipedia.org/wiki/Energy_density}{energy
  density} of \href{http://en.wikipedia.org/wiki/Petroleum}{crude oil}
$E_{\rm oil}$ compare with the energy density of hot
\href{http://en.wikipedia.org/wiki/Geothermal_energy}{geothermal}
fluid $E_{\rm geo}$?

Model the geothermal fluid as water, and assume its starts at 360 K
when pumped from oil/geothermal wells at a rate of 30 l/s, and is
cooled to 300 K.

Give the ratio of energy densities.

\edXinline{$E_{\rm oil}/E_{\rm geo}$ =} \edXabox{expect="177" type="numerical"
  tolerance='20\%' inline='1' hintfn="hint_mag0" } 
 
\solution{
  If the geothermal fluid is basically just water, then its density at
  360~K is approximately 1~g/cm$^3$ (really is 0.97~g/cm$^3$ but also
  changes with temperature as heat is extracted). The heat capacity is
  approximately $c = 4.18~kJ/kg K$. If the fluid is cooled to a
  typical temperature of 300~K, then the total energy density of the
  geothermal fluid is
\begin{equation}
\mathcal{E}_{\rm geo} = \rm(4.18~kJ/kg~K)(60~K) = 251~kJ/kg
\end{equation}
Oil has an energy density of around 43~MJ/kg (wikipedia gives 46
MJ/kg).  Thus, $E_{\rm oil}/E_{\rm geo} \approx 177$, so oil has
nearly 200 times the energy density of the geothermal fluid.  }

\end{edXproblem}

%%%%%%%%%%%%%%%%%%%%%%%%%%%%%%%%%%%%%%%%
\begin{edXproblem}{Staff Chapter 33 problem 13b}{attempts=5}

\begin{edXscript}
expected=[ 0.078, 4.6e-4 ]
\end{edXscript}

\edXinclude{XML/magnitude_hint2.xml}

Both oil and geothermal fluid need to be pumped from the ground.
Compare the power needed to pump the geothermal fluid $P_{\rm geo, grav}$ with the power
available from the geothermal fluid $P_{\rm geo}$ (when pumped at 30
l/s).

Assume the pump provides all energy needed to lift the fluid from a
depth of 2~km.  Disregard the lift from reservoir pressure.

\edXinline{$P_{\rm geo, grav}/P_{\rm geo}$ =} \edXabox{expect="0.078" type="numerical"
  tolerance='5\%' inline='1' hintfn="hint_mag0" } 

Now do similarly for oil.  
Compare the power needed to pump oil $P_{\rm oil, grav}$ with the power
available from the oil $P_{\rm oil}$ (when pumped at 30
l/s).  Assume the oil has a mass density of $0.9$ kg/l.

\edXinline{$P_{\rm oil, grav}/P_{\rm  oil}$ =} \edXabox{expect="4.6e-4" type="numerical"
  tolerance='12\%' inline='1' hintfn="hint_mag1" } 

\solution{
At 30~l/s, the total power provided by the geothermal source is
\begin{equation}
P_{\rm geo} = E_{\rm geo}\rm(30~l/s)(1~kg/l)  = 7.5~MW
\end{equation}
The total rate of energy consumption due to gravitational potential energy is 
\begin{equation}
P_{\rm geo, grav} = \rho \ell g Av
\end{equation}
where $\ell = 2$ km is the depth and $Av$ is the flow rate (30~l/s). For
the geothermal well, $P_{\rm geo, grav} = 588$~kW.  Thus
\begin{equation}
  P_{\rm geo, grav}/P_{\rm geo} = 0.078
\,,
\end{equation}
or less than 10\% of the
total energy available. However, once the efficiency of the pump is
considered, a large fraction of the energy from the well would need to
be used just to run the pump.

The mass density of oil is approximately 0.9~kg/l, so for the oil
well, energy contained in the oil is equivalent to
\begin{equation}
P_{\rm oil} = (43~MJ/kg)(30~l/s)(0.9~kg/l) = 1.16~GW
\end{equation}
The power required to pump the oil from a depth of 2~km is just
$P_{\rm oil, grav}=529$~kW, so
\begin{equation}
  P_{\rm geo, grav}/P_{\rm geo} \approx 4.6\times 10^{-4}
\,,
\end{equation}
which is very small compared to the amount of energy that can be extracted from the oil.
}

\end{edXproblem}
%%%%%%%%%%%%%%%%%%%%%%%%%%%%%%%%%%%%%%%%

\end{edXvertical}
 

\end{edXsection}

%%%%%%%%%%%%%%%%%%%%%%%%%%%%%%%%%%%%%%%%%%%%%%%%%%%%%%%%%%%%%%%%%%%%%%%%%%%%%


\begin{edXsection}{Staff Grading Tests}

\edXinclude{XML/staff_grading1.xml}

\end{edXsection}

%%%%%%%%%%%%%%%%%%%%%%%%%%%%%%%%%%%%%%%%%%%%%%%%%%%%%%%%%%%%%%%%%%%%%%%%%%%%%

%%%%%%%%%%%%%%%%%%%%%%%%%%%%%%%%%%%%%%%%%%%%%%%%%%%%%%%%%%%%%%%%%%%%%%%%%%%%%


\begin{edXsection}{Staff Problems II}

\begin{edXvertical}

\begin{edXproblem}{Test operator equations}

Let $X$, $Y$, and $Z$ be the usual Pauli matrices,
\bea
     	X &=& \mattwoc{0}{1}{1}{0}
\\ 	Y &=& \mattwoc{0}{-i}{i}{0}
\\ 	Z &=& \mattwoc{1}{0}{0}{-1}
\,.
\eea
Give an expression equal to $iZ$.

\begin{edXscript}

from matrix_evaluator import *

\end{edXscript}

\edXinline{$|\lambda\> = $ }
\edXabox{expect="X*Y" 
  type="custom" cfn="test_formula" 
  inline='1'
  math="1" 
  size="70"
  options="samples='X,Y,Z,i@[0|1;1|0],[0|(0-1j);(0+1j)|0],[1|0;0|-1],0+1j:[0|1;1|0],[0|(0-1j);(0+1j)|0]],[1|0;0|-1],0+1j#50'!altanswer='-Y*X'!altanswer='i*Z'"
  preprocessorClassName="MathjaxPreprocessorForQM" preprocessorSrc="/static/js/mathjax_preprocessor_for_QM.js"
 }%

  % options="samples='X,Y,i@[1|2;3|4],[0|2;4|6],0+1j:[5|5;5|5],[8|8;8|8],0+1j#50'!altanswer='-Y*X'!altanswer='i*Z'"

\begin{edXsolution}

Either $XY$ or $-YX$ but not $YX$.

\end{edXsolution}

\end{edXproblem}

\end{edXvertical}

\end{edXsection}

%%%%%%%%%%%%%%%%%%%%%%%%%%%%%%%%%%%%%%%%%%%%%%%%%%%%%%%%%%%%%%%%%%%%%%%%%%%%%

\end{edXchapter}

%%%%%%%%%%%%%%%%%%%%%%%%%%%%%%%%%%%%%%%%%%%%%%%%%%%%%%%%%%%%%%%%%%%%%%%%%%%%%

